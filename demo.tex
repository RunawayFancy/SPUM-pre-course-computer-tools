\documentclass[UTF8,a4paper,15pt,titlepage,scale=0.8]{article}

\usepackage[a4paper]{geometry}
\usepackage{fancyhdr}
\usepackage{amsmath}
\usepackage{graphicx}
\usepackage{color}
\usepackage{enumerate}
\usepackage{amssymb}
\usepackage{siunitx}
\usepackage{indentfirst}
\usepackage{pgfplots}
\usepackage{graphicx}
\usepackage{float}
\usepackage{subfigure}
\usepackage{mathtools}
\usepackage{mhchem}
\usepackage{tensor}
\usepackage{gensymb}
%
\usepackage{physics}
%
% \usepackage{tikz}
\usepackage{makecell}

\pgfplotsset{width=10cm,height=7cm,compat=1.13}

\geometry{left=2.0cm,right=2.0cm,top=2.0cm,bottom=2.0cm}

\definecolor{grey}{rgb}{0.85,0.85,0.85}

\setlength{\parindent}{2em}

\begin{document}
    
\tableofcontents
\begin{equation}
    \ket*{\psi} = \mqty[1\\ 0]
\end{equation}
\textbf{apple}

\textit{pen} \underline{pinnaple}

\begin{equation}
    qsres \  atata
\end{equation}
$$\ket{\psi}$$ is a quantum state
\begin{equation}
    \alpha  \beta \gamma
\end{equation}
\begin{equation}
    \Sigma \Gamma \Theta \Xi \Delta 
\end{equation}
\begin{equation}
    \le \ge \ll \neq \equiv 
\end{equation}
\begin{equation}
    \forall \exists \in \subseteq \subset 
\end{equation}
\begin{equation}
    \sum_{i=1}^{20} a_i
\end{equation}
\begin{equation}
    \int_1 ^2 f(x) \dd x
\end{equation}
\begin{equation}
    \sin (x) \ \ln(x) \ \lim_{\Delta x\rightarrow 0} \frac{f(x+\Delta x) - f(x)}{\Delta x }
\end{equation}
\begin{equation}
    \rightarrow \leftarrow \Rightarrow \Leftarrow
\end{equation}
\begin{equation}
    \rightleftarrows \leftrightarrow \Leftrightarrow \Downarrow 
\end{equation}
\begin{equation}
    \otimes \oplus \times \cdot \cdots \vdots 
\end{equation}
\begin{equation}
    \dd \ \partial \ \mathbb{R} \ \mathbb{Z} \ \mathbb{C}
\end{equation}
\begin{equation}
    \frac{a}{b} \ \frac{\partial F}{\partial t}
\end{equation}
\begin{equation}
    \begin{aligned}
        y_1 &= f_1(x)\\
        y_2 &= f_2(x) + g_2(x)\\
        y_3 &= f_3(x) + g_3(x) + \psi(x)
    \end{aligned}
\end{equation}
\begin{equation}
    \mqty[1&2&3 \\ 4&5&6]
\end{equation}
\begin{equation}
    f(x) = 
    \begin{cases}
        x  \ \ x>0 \\
        -1 \ \ x=0\\
        -x  \ \ x<0
    \end{cases}
\end{equation}



\end{document}