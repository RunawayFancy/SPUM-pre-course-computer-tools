\documentclass[UTF8,a4paper,15pt,titlepage,scale=0.8]{article}%outside可以消除偶数页留白, 这里的页面设置也可以用\geometry{a4paper,scale=0.8}
%\usepackage[zihao=-4]{ctex} % ctex可以写中文,中括号里那句的意思是正文小四号
\usepackage[a4paper]{geometry} % 调整纸张大小和页边距的包,这里中括号中规定了纸张大小
\usepackage{fancyhdr}
\usepackage{amsmath}
\usepackage{graphicx}% 用它在报告里加图
\usepackage{color}
\usepackage{enumerate}
\usepackage{amssymb}
\usepackage{siunitx}%封面表格
\usepackage{indentfirst}%第一段的首行缩进
\usepackage{pgfplots}% 使用pgfplots绘图工具包
\usepackage{graphicx} %插入图片的宏包
\usepackage{float} %设置图片浮动位置的宏包
\usepackage{subfigure} %插入多图时用子图显示的宏包
\usepackage{mathtools}
\usepackage{mhchem}
\usepackage{tensor}
\usepackage{gensymb}
\usepackage{physics}
\usepackage[colorlinks,linkcolor=blue]{hyperref}


\pgfplotsset{width=10cm,height=7cm,compat=1.13} % 图片绘制的宽度是7cm,使用的pgfplots版本为1.13

%\geometry{left=2.0cm,right=2.0cm,top=2.0cm,bottom=2.0cm} % 页边距设置

\definecolor{grey}{rgb}{0.85,0.85,0.85}

\graphicspath{Pictures/}%设置图片路径为当前路径下的pictures文件夹


%\setcounter{chapter}{5}%重新计数章节
%\numberwithin{equation}{chapter}%公式排序

\setlength{\parindent}{2em}%统一取消首行缩进

\begin{document}
\begin{center}
    \large{\textbf{SPUM 101}} \\
    \textit{Course: Classical mechanics}
\end{center}

\section*{Introduction to this lecture}
This lecture is an experimental lecture to estimate the value of the existence of the SLUM including can participants really have a passion for physics and physics studying, the quality of our lectures, how much could our participants learn from our lecture and the hard level of the concepts. Your feedback is really important for us to provide high-quality educational resources. The pre-lecture will include the following concepts.
\paragraph{Concepts}
\begin{enumerate}[$\cdot$]
    \item Introduction to variation
    \item Hamilton principle(The least action principle)
    \item Euler-Lagrange equation of motion
    \item Virtual force \& work and D'Alembert's principle
    \item Conservation law and symmetry
    \item Integration of the equation of motion
    \item Harmonic oscillation
    \item Rigid body
    \item Euler angles and equations of motion
    \item Hamilton mechanics
\end{enumerate}
\paragraph{Recommended textbooks}
\begin{enumerate}[$\cdot$]
    \item Landau, L. D., Lifshits, E. M., \& Lifsic, E. M. (1960). Mechanics (Vol. 1). CUP Archive.
    \item Lanczos, C. (2020). The variational principles of mechanics. University of Toronto press.
\end{enumerate}

\paragraph{Time}
19:00 - 21:00, Wed

\paragraph{Location}
E11-1038

\paragraph{Instructor} 

Gigi Duan 

profile link: runawayfancy.me

e-mail: gigi.duan@connect.um.edu.mo
\end{document}